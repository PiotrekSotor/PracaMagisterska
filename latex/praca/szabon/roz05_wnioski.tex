\newpage
\chapter{Wnioski i podsumowanie}
\begin{itemize}
%\item Analiza porównawcza:
%\begin{itemize}
%\item dla kryt nr 1 - algorytm memetyczny i genetyczny uzyskał bardzo podobne rozwiązania, ale memetyczny znacznie szybciej
%\item dla kryt nr 1 - mutGauss opóźnił stagnację memetycznego i przyspieszył genetycznego
%\item dla kryt nr 1 - PSO dużo gorszy od pozostałych
%\item dla kryt nr 2 - memetyczny z mutGauss najszybciej osiągnął kolejne progi progi
%\item dla kryt nr 2 - mutGauss wyraźnie lepszy od mutDefault
%\item dla kryt nr 2 - PSO dużo gorszy
%\item dla kryt nr 3 - memetyczny z mutGauss zdecydowanie najlepszy dla Schaffer nr 2, dla pozostałych równy z memetycznym z mutDefault
%\item dla kryt nr 3 - mutGauss lepszy niż mutDefault
%\item dla kryt nr 4 - największy wpływ na różnicę ma operator mutacji
%\item dla kryt nr 4 - mem i gen z tymi samymi operatorami osiągaja bardzo podobne rezultaty
%\item dla kryt nr 4 - największa dynamika zmian różnicy jest w pierwszych 20 iteracjach
%
%\item
%\end{itemize} 
\item Dobór parametrów:
\begin{itemize}
\item dla funkcji trudnooptymalizowanych najlepsze konfiguracje uzyskano scenariuszami dobierającymi trójkami
\item nieznacznie gorsze okazał sie scenariusz [3-6, 1-2, 4-5]
\item dobieranie dwójkami jest znacznie szybsze
\item mutGauss uzyskuje lepsze rezultaty, ale proces selekcji jest wolniejszy
\end{itemize}
\item powstała aplikacja umożliwia powtórzenie i przeprowadzenie kolejnych badan
\end{itemize}

\par
Analizując wyniki uzyskane w trakcie badań porównawczych można dojść do wniosku, że algorytm memetyczny stanowi znacznie lepszą metodę rozwiązywania rzeczywistoliczbowych problemów optymalizacyjnych. Prawie dla każdego z wykorzystanych kryteriów uzyskał znacznie lepsze rezultaty niż pozostałe algorytmy wykorzystane do porównania. Ostateczne podsumowanie rezultatów uzyskanych dla użytych kryteriów rozpatrując głównie różnice pomiędzy wykorzystywanymi algorytmami przedstawia się następująco:
\begin{itemize}
\item kryterium nr 1 wykazało, że zarówno algorytm memetyczny jak i genetyczny uzyskuje bardzo podobne, bliskie optimum rozwiązania, ale memetyczny dokonuje tego szybciej, 
\item kryterium nr 2 wykazało, że algorytm memetyczny wykorzystujący mutację \emph{mutGauss} najszybciej osiągnął kolejne progi jakości rozwiązania, 
\item kryterium nr 3 wykazało przewagę w powtarzalności uzyskiwania rezultatów bliskich optimum nad algorytmem genetycznym i \emph{PSO},
\item kryterium nr 4 nie wykazało większej zależności pomiędzy użytym algorytmem, a różnicą pomiędzy najlepszym i średnim rozwiązaniem w populacji - wpływ na ten aspekt ma wykorzystany operator mutacji.
\end{itemize}

\par 
Przeprowadzenie badań z wykorzystaniem dwóch wersji operatora mutacji pozwoliło na przeprowadzenie analizy różnicy w ich skuteczności. W ogólnym rozrachunku lepsze rezultaty uzyskano z wykorzystaniem \emph{mutGauss}. Kosztem uzyskania wyższych wyników przy wykorzystaniu tego operatora jest jednak dłuższy czas działania pojedynczej iteracji algorytmu. Może to być jednakże konsekwencja nieoptymalnej własnej implementacji spowodowanej niewystarczającą biegłością autora pracy w posługiwaniu się językiem \emph{R}. Zebranie wniosków dotyczących porównania dwóch wykorzystanych operatorów mutacji można opisano poniżej:
\begin{itemize}
\item kryterium nr 1 wskazało na to, że użycie \emph{mutGauss} przyspiesza stagnację populacji algorytmu memetycznego, ale opóźnia w przypadku algorytmu genetycznego,
\item kryterium nr 2 wykazało zdecydowaną przewagę operatora \emph{mutGauss} nad \emph{mutDefault} jeśli chodzi o szybkość osiągania poszczególnych progów jakości rozwiązania
\item kryterium nr 3 wskazało na to, że mutacja gaussowska umożliwie z większą powtarzalnością uzyskanie rozwiązania bliskiego optimum,
\item kryterium nr 4 wykazało, że zastosowanie domyślnego operatora mutacji znacznie zwiększa różnorodność genetyczna populacji, czego oznaką jest większa różnica pomiędzy najlepszym i średnim rozwiązaniem populacji.
\end{itemize}

\par
Pełne podsumowanie badań porównawczych przedstawiono w tabeli \ref{table:wnioski_zestawienie_wynikow}. Znakiem $+$ oznaczono tam, który z algorytmów i operatorów uzyskał najlepsze rezultaty dla danego kryterium i funkcji testowej. Oznaczenie dwóch algorytmów lub mechanizmów mutacji dla jednego kryterium jest konsekwencją uzyskania tych samych wyników przez więcej niż jedną badaną konfigurację. Ostatni wiersz tabeli przedstawiający sumę \emph{zwycięstw} poszczególnych opcji ostatecznie potwierdza przewagę algorytmu memetycznego nad pozostałymi wykorzystanymi w analizie porównawczej. Niewielka różnica w wynikach dla operatorów mutacji spowodowana jest faktem, że dla wielu kryteriów wyniki uzyskane obydwoma były takie same, przypadki zwycięstwa \emph{mutDefault} uzyskane zostały przy niewielkiej różnicy względem \emph{mutGauss}, zaś każda z sytuacji dominacji gaussowskiego mechanizmu mutowania osobników odznaczała się zdecydowaną przewagą nad domyślnym algorytmem pakietu \emph{GA}.
 

\newcolumntype{?}{@{\hskip\tabcolsep\vrule width 1pt\hskip\tabcolsep}}

\begin{table}[hbt]
\caption{Zestawienie wyników badań dla kryteriów porównawczych}
\label{table:wnioski_zestawienie_wynikow}
\begin{center}
\begin{tabularwithnotes}{|c|l?c|c|c?c|c|}
{
  \tnote[1]{\emph{Algorytm memetyczny}}
  \tnote[2]{\emph{Algorytm genetyczny}}
 }
	\Xhline{1pt}
	\multirow{2}{*}{Funkcja} & \multicolumn{1}{c?}{\multirow{2}{*}{Kryterium}} & \multicolumn{3}{c?}{Algorytm} & \multicolumn{2}{c|}{Operator mutacji} \\
	\cline{3-7}	
	{} & {} & \emph{mem}\tmark[1] & \emph{gen}\tmark[2] & \emph{PSO} & \emph{mutGauss} & \emph{mutDefault} \\
	\Xhline{1pt}	
	\multirow{8}{*}{\emph{Schaffer nr 2}} & Kr. 1 - rozwiązanie &+ & & &+ & \\ \cline{2-7}
	{} & Kr. 1 - stagnacja &+ & & & &+ \\ \cline{2-7}
	{} & Kr. 2 - próg 99\% &+ & & &+ & \\ \cline{2-7}
	{} & Kr. 2 - próg 95\% & &+ & &+ & \\ \cline{2-7}
	{} & Kr. 3 - dokładność 0.005 &+ & & &+ & \\ \cline{2-7}
	{} & Kr. 3 - dokładność 0.01 &+ & & &+ & \\ \cline{2-7}
	{} & Kr. 4 - różnica po 10 it. & &+ & &+ & \\ \cline{2-7}
	{} & Kr. 4 - różnica po 100 it. & & &+ &+ & \\
	\Xhline{1pt}
	\multirow{8}{*}{\emph{Paviani}} & Kr. 1 - rozwiązanie &+ & & &+ &+ \\ \cline{2-7}
	{} & Kr. 1 - stagnacja &+ & & & &+ \\  \cline{2-7}
	{} & Kr. 2 - próg 99\% &+ & & &+ &+ \\ \cline{2-7}
	{} & Kr. 2 - próg 95\% &+ & & &+ &+ \\ \cline{2-7}
	{} & Kr. 3 - dokładność 0.005 &+ & & &+ &+ \\ \cline{2-7}
	{} & Kr. 3 - dokładność 0.01 &+ & & &+ &+ \\ \cline{2-7}
	{} & Kr. 4 - różnica po 10 it. &+ & & &+ & \\ \cline{2-7}
	{} & Kr. 4 - różnica po 100 it. & &+ &+ & &+ \\
	\Xhline{1pt}
	\multirow{8}{*}{\emph{ZeldaSine10}} & Kr. 1 - rozwiązanie &+ &+ & &+ &+ \\ \cline{2-7}
	{} & Kr. 1 - stagnacja &+ & & & &+ \\ \cline{2-7}
	{} & Kr. 2 - próg 99\% &+ & & &+ & \\ \cline{2-7}
	{} & Kr. 2 - próg 95\% &+ & & &+ & \\ \cline{2-7}
	{} & Kr. 3 - dokładność 0.005 &+ & & & &+ \\ \cline{2-7}
	{} & Kr. 3 - dokładność 0.01 &+ & & & &+ \\ \cline{2-7}
	{} & Kr. 4 - różnica po 10 it. &+ & & & &+ \\ \cline{2-7}
	{} & Kr. 4 - różnica po 100 it. &+ &+ & & &+ \\
	\Xhline{1pt}
	\multicolumn{2}{|c?}{Suma} & 20 & 5 & 2 & 16 & 14 \\
	\Xhline{1pt}
\end{tabularwithnotes}
\end{center}
\end{table}


