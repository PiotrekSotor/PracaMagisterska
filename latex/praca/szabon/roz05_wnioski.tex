\newpage
\chapter{Wnioski i podsumowanie}
%\begin{itemize}
%\item Analiza porównawcza:
%\begin{itemize}
%\item dla kryt nr 1 - algorytm memetyczny i genetyczny uzyskał bardzo podobne rozwiązania, ale memetyczny znacznie szybciej
%\item dla kryt nr 1 - mutGauss opóźnił stagnację memetycznego i przyspieszył genetycznego
%\item dla kryt nr 1 - PSO dużo gorszy od pozostałych
%\item dla kryt nr 2 - memetyczny z mutGauss najszybciej osiągnął kolejne progi progi
%\item dla kryt nr 2 - mutGauss wyraźnie lepszy od mutDefault
%\item dla kryt nr 2 - PSO dużo gorszy
%\item dla kryt nr 3 - memetyczny z mutGauss zdecydowanie najlepszy dla Schaffer nr 2, dla pozostałych równy z memetycznym z mutDefault
%\item dla kryt nr 3 - mutGauss lepszy niż mutDefault
%\item dla kryt nr 4 - największy wpływ na różnicę ma operator mutacji
%\item dla kryt nr 4 - mem i gen z tymi samymi operatorami osiągaja bardzo podobne rezultaty
%\item dla kryt nr 4 - największa dynamika zmian różnicy jest w pierwszych 20 iteracjach
%
%\item
%\end{itemize} 
%\item Dobór parametrów:
%\begin{itemize}
%\item dla funkcji trudnooptymalizowanych najlepsze konfiguracje uzyskano scenariuszami dobierającymi trójkami
%\item nieznacznie gorsze okazał sie scenariusz [3-6, 1-2, 4-5]
%\item dobieranie dwójkami jest znacznie szybsze
%\item mutGauss uzyskuje lepsze rezultaty, ale proces selekcji jest wolniejszy
%\end{itemize}
%\item powstała aplikacja umożliwia powtórzenie i przeprowadzenie kolejnych badan
%\item
%\end{itemize}

\par
Analizując wyniki uzyskane w trakcie badań porównawczych można dojść do wniosku, że algorytm memetyczny stanowi bardzo dobrą metodę rozwiązywania rzeczywistoliczbowych problemów optymalizacyjnych. Prawie dla każdego z wykorzystanych kryteriów uzyskał znacznie lepsze rezultaty niż pozostałe algorytmy. Podsumowując wyniki uzyskane dla użytych kryteriów, różnice pomiędzy wykorzystywanymi algorytmami przedstawiają się następująco:
\begin{itemize}
\item badanie kryterium nr 1 wykazało, że zarówno algorytm memetyczny jak i genetyczny uzyskuje bardzo podobne, bliskie optimum rozwiązania, ale memetyczny dokonuje tego szybciej, 
\item analiza kryterium nr 2 wykazała, że algorytm memetyczny wykorzystujący mutację \emph{mutGauss} najszybciej osiągnął kolejne progi jakości rozwiązania, 
\item badanie kryterium nr 3 wykazało przewagę w powtarzalności uzyskiwania rezultatów bliskich optimum algorytmu memetycznego nad genetycznym i \emph{PSO},
\item analiza kryterium nr 4 nie wykazała większej zależności pomiędzy użytym algorytmem, a różnicą pomiędzy najlepszym i średnim rozwiązaniem w populacji - wpływ na ten aspekt ma wykorzystany operator mutacji.
\end{itemize}

\par 
Przeprowadzenie badań z wykorzystaniem dwóch wersji operatora mutacji pozwoliło na przeprowadzenie analizy różnicy w ich skuteczności. W ogólnym rozrachunku lepsze rezultaty uzyskano z wykorzystaniem \emph{mutGauss} kosztem dłuższego czasu działania pojedynczej iteracji algorytmu. Może to być jednakże konsekwencją nieoptymalnej własnej implementacji spowodowanej niewystarczającą biegłością autora pracy w posługiwaniu się językiem \emph{R}. Wnioski dotyczące porównania dwóch wykorzystanych operatorów mutacji przedstawiono poniżej:
\begin{itemize}
\item badanie kryterium nr 1 wskazało na to, że użycie \emph{mutGauss} przyspiesza stagnację populacji algorytmu memetycznego, ale opóźnia ją w przypadku algorytmu genetycznego,
\item analiza kryterium nr 2 wykazała zdecydowaną przewagę operatora \emph{mutGauss} nad \emph{mutDefault} jeśli chodzi o szybkość osiągania poszczególnych progów jakości rozwiązania,
\item badanie kryterium nr 3 wskazało na to, że mutacja gaussowska umożliwia z większą powtarzalnością uzyskanie rozwiązania bliskiego optimum,
\item analiza kryterium nr 4 wykazała, że zastosowanie domyślnego operatora mutacji znacznie zwiększa różnorodność genetyczną populacji, czego oznaką jest większa różnica pomiędzy najlepszym i średnim rozwiązaniem populacji.
\end{itemize}

\par
Pełne podsumowanie badań porównawczych przedstawiono w tabeli \ref{table:wnioski_zestawienie_wynikow}. Znakiem ''$+$'' oznaczono tam, który z algorytmów i operatorów uzyskał najlepsze rezultaty dla danego kryterium i funkcji testowej. Oznaczenie dwóch algorytmów lub mechanizmów mutacji dla jednego kryterium jest konsekwencją uzyskania tych samych wyników przez więcej niż jedną badaną konfigurację. Ostatni wiersz tabeli przedstawiający sumę \emph{zwycięstw} poszczególnych opcji ostatecznie potwierdza przewagę algorytmu memetycznego nad pozostałymi wykorzystanymi w analizie porównawczej. Niewielka różnica w wynikach dla operatorów mutacji spowodowana jest faktem, że dla wielu kryteriów wyniki uzyskane obydwoma operatorami były takie same. Przypadki lepszych rezultatów \emph{mutDefault} uzyskane zostały przy niewielkiej różnicy względem \emph{mutGauss}, zaś każda z sytuacji dominacji gaussowskiego mechanizmu mutowania osobników odznaczała się zdecydowaną przewagą nad domyślnym operatorem pakietu \emph{GA}.
 

\newcolumntype{?}{@{\hskip\tabcolsep\vrule width 1pt\hskip\tabcolsep}}

\begin{table}[hbt]
\caption{Zestawienie wyników badań dla kryteriów porównawczych}
\label{table:wnioski_zestawienie_wynikow}
\begin{center}
\begin{tabularwithnotes}{|c|l?c|c|c?c|c|}
{
  \tnote[1]{\emph{Algorytm memetyczny}}
  \tnote[2]{\emph{Algorytm genetyczny}}
 }
	\Xhline{1pt}
	\multirow{2}{*}{Funkcja} & \multicolumn{1}{c?}{\multirow{2}{*}{Kryterium}} & \multicolumn{3}{c?}{Algorytm} & \multicolumn{2}{c|}{Operator mutacji} \\
	\cline{3-7}	
	{} & {} & \emph{mem}\tmark[1] & \emph{gen}\tmark[2] & \emph{PSO} & \emph{mutGauss} & \emph{mutDefault} \\
	\Xhline{1pt}	
	\multirow{8}{*}{\emph{Schaffer nr 2}} & Kr. 1 - rozwiązanie &+ & & &+ & \\ \cline{2-7}
	{} & Kr. 1 - stagnacja &+ & & & &+ \\ \cline{2-7}
	{} & Kr. 2 - próg 99\% &+ & & &+ & \\ \cline{2-7}
	{} & Kr. 2 - próg 95\% & &+ & &+ & \\ \cline{2-7}
	{} & Kr. 3 - dokładność 0.005 &+ & & &+ & \\ \cline{2-7}
	{} & Kr. 3 - dokładność 0.01 &+ & & &+ & \\ \cline{2-7}
	{} & Kr. 4 - różnica po 10 it. & &+ & &+ & \\ \cline{2-7}
	{} & Kr. 4 - różnica po 100 it. & & &+ &+ & \\
	\Xhline{1pt}
	\multirow{8}{*}{\emph{Paviani}} & Kr. 1 - rozwiązanie &+ & & &+ &+ \\ \cline{2-7}
	{} & Kr. 1 - stagnacja &+ & & & &+ \\  \cline{2-7}
	{} & Kr. 2 - próg 99\% &+ & & &+ &+ \\ \cline{2-7}
	{} & Kr. 2 - próg 95\% &+ & & &+ &+ \\ \cline{2-7}
	{} & Kr. 3 - dokładność 0.005 &+ & & &+ &+ \\ \cline{2-7}
	{} & Kr. 3 - dokładność 0.01 &+ & & &+ &+ \\ \cline{2-7}
	{} & Kr. 4 - różnica po 10 it. &+ & & &+ & \\ \cline{2-7}
	{} & Kr. 4 - różnica po 100 it. & &+ &+ & &+ \\
	\Xhline{1pt}
	\multirow{8}{*}{\emph{ZeldaSine10}} & Kr. 1 - rozwiązanie &+ &+ & &+ &+ \\ \cline{2-7}
	{} & Kr. 1 - stagnacja &+ & & & &+ \\ \cline{2-7}
	{} & Kr. 2 - próg 99\% &+ & & &+ & \\ \cline{2-7}
	{} & Kr. 2 - próg 95\% &+ & & &+ & \\ \cline{2-7}
	{} & Kr. 3 - dokładność 0.005 &+ & & & &+ \\ \cline{2-7}
	{} & Kr. 3 - dokładność 0.01 &+ & & & &+ \\ \cline{2-7}
	{} & Kr. 4 - różnica po 10 it. &+ & & & &+ \\ \cline{2-7}
	{} & Kr. 4 - różnica po 100 it. &+ &+ & & &+ \\
	\Xhline{1pt}
	\multicolumn{2}{|c?}{Suma} & 20 & 5 & 2 & 16 & 14 \\
	\Xhline{1pt}
\end{tabularwithnotes}
\end{center}
\end{table}

\par
Przy okazji pracy nad analizą efektywności algorytmów memetycznych sprawdzono autorską koncepcję sposobu doboru jego konfiguracji. Zaproponowano i przetestowano przy wykorzystaniu funkcji testowych scenariusze cechujące się znacznie krótszym czasem wykonania niż tradycyjne podejście sprawdzające wszystkie możliwe kombinacje wartości atrybutów. Uzyskane w ten sposób parametry algorytmów wystarczały, by wykazać zdecydowaną przewagę algorytmów memetycznych nad \emph{algorytmami genetycznymi} i \emph{PSO} nawet pomimo faktu, że konfiguracje obu z nich dobierane były poprzez pełny przegląd wszystkich kombinacji. Analiza wyników procesu selekcji parametrów na osi wykorzystanego operatora mutacji potwierdza już przytoczoną obserwację dotyczącą negatywnego wpływu stosowania \emph{mutGauss} na czas wykonania algorytmu.
\par
Tworząc koncepcję scenariuszy doboru parametrów zdecydowano się na sprawdzenie dwóch podejść - selekcji dwóch lub trzech parametrów jednocześnie. Pierwszy z wariantów cechuje się bardzo niskim czasem wykonania, a jakość w ten sposób uzyskanych konfiguracji jest tylko w nieznacznym stopniu gorsza od drugiego podejścia. W przypadku funkcji łatwiejszych z punktu widzenia optymalizacji szybkie scenariusze selekcji parami okazują się być wystarczające, czego dowodem jest wybór scenariusza $[4-5,\;3-6,\;1-2]$ w przypadku funkcji \emph{ZeldaSine10}.
\par
Aplikacja powstała w trakcie pracy nad analizą efektywności algorytmów memetycznych umożliwia powtórzenie wszystkich przeprowadzonych badań dla innych, definiowanych przez użytkownika rzeczywistoliczbowych problemów optymalizacyjnych. Dodatkowo, poprzez modyfikację skryptów przez nią wykorzystywanych można manipulować procesem zarówno selekcji parametrów algorytmów, jak i szczegółami związanymi z kryteriami porównawczymi (np. zmiana dokładności pomiaru lub maksymalnego czasu wykonania dla \emph{kryterium nr 2}). 
\par
W trakcie pracy zrealizowano wszystkie założone cele. Wykazano, że algorytm memetyczny dostępny w ramach pakietu \emph{GA} dla języka \emph{R} stanowi bardziej efektywny mechanizm rozwiązywania rzeczywistoliczbowych problemów optymalizacyjnych niż pozostałe badane mechanizmy. Zaczerpnięty z literatury operator mutacji gaussowskiej okazał się znacznie lepszą metodą niż domyślnie wykorzystana przez autora pakietu. Zaproponowane podejście określania konfiguracji pracy algorytmu memetycznego można uznać za skuteczne, szczególnie jeśli ważniejszym aspektem jest czas doboru parametrów niż uzyskanie optymalnej jakości działania.



