\newpage
\chapter{Badania}
\textbf{Przerzucone z wcześniejszych rozdziałów:}
\begin{itemize}
\item W przypadku wszystkich opisanych w tym rozdziale algorytmów za \emph{warunkiem stopu} można rozumieć wykonanie z góry określonej maksymalnej liczby iteracji pętli lub brak poprawy najlepszego rozwiązania w pewnej liczbie ostatnich przebiegów algorytmu. 
\item Do ustawiana elitismu w genetycznym i memetycznym - Tak również postąpiono w trakcie przeprowadzania badań w ramach tej pracy - wykorzystano domyślny w pakiecie \emph{GA} rozmiar grupy chronionych osobników wynoszący 5\% populacji.
\end{itemize}


\section{Specyfikacja funkcji testowych}

\section{Opis środowiska testowego}

\section{Przyjęte miary efektywności algorytmów}
\label{sec:przyjete_miary_efektywnosci_algorytmow}