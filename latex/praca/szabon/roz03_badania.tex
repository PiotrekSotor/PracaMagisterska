\newpage
\chapter{Badania}
%\textbf{Przerzucone z wcześniejszych rozdziałów:}
%\begin{itemize}
%\item W przypadku wszystkich opisanych w tym rozdziale algorytmów za \emph{warunkiem stopu} można rozumieć wykonanie z góry określonej maksymalnej liczby iteracji pętli lub brak poprawy najlepszego rozwiązania w pewnej liczbie ostatnich przebiegów algorytmu. 
%\item Do ustawiana elitismu w genetycznym i memetycznym - Tak również postąpiono w trakcie przeprowadzania badań w ramach tej pracy - wykorzystano domyślny w pakiecie \emph{GA} rozmiar grupy chronionych osobników wynoszący 5\% populacji.
%\end{itemize}


\section{Specyfikacja funkcji testowych}
Znajdowanie coraz lepszych metod rozwiązywania problemów optymalizacyjnych oraz ich testowanie jest zagadnieniem na tyle powszechnym, że powstały w tym celu narzędzia to ułatwiające. Dzięki ujednoliceniu sposobu badań wydajności algorytmów możliwe jest porównywanie ich na płaszczyźnie tych samych problemów. Jednym z pakietów języka \emph{R} powstałym w tym celu jest \emph{globalOptTests} \cite{globalOptTestsPackage}. Stanowi go zbiór kilkunastu funkcji o różnej liczbie parametrów. Każda z nich dodatkowo opisana jest dziedziną wartości poszczególnych argumentów oraz wartością funkcji w ekstremum. Wszystkie problemy zawarte w pakiecie są zadaniami znalezienia \emph{minimum} globalnego.
\par
Na potrzeby przeprowadzenia badań porównawczych różnych algorytmów wybrano z pakiety \emph{globalOptTests} 3 funkcje, które zostaną opisane poniżej. Reprezentują one problemy  optymalizacyjne o różnej wymiarowości. 
\subsection{Funkcja Schaffer1}
\begin{itemize}
\item wzor
\item wykres, bo 2D
\item dziedzina argumentow
\item optimum: 0, w początku układu współrzędnych
\item cechy: wiele ekstremow lokalnych zblizonych do optimum, ale odseparowanych. Większość dziedziny rozwiązań jest płaska jak stół 
\end{itemize}
\subsection{Funkcja Paviani}
\begin{itemize}
\item wzor
\item dziedzina argumentow
\item optimum: jakie i gdzie
\item pewnie cos w necie ciekawego o nim jest
\end{itemize}

\section{Opis środowiska testowego}
\begin{itemize}
\item parametry maszyny testowej

\end{itemize}


\section{Przyjęte miary efektywności algorytmów}
\label{sec:przyjete_miary_efektywnosci_algorytmow}
wszstko dla tych samych liczby iteracji
\begin{enumerate}
\item najlepsze znalezione rozwiązanie kolejno w iteracjach
\item średnie rozwiązanie populacji kolejno w iteracjach
\item czas osiągnięcia rozwiązań 90, 95 i 99%

\end{enumerate}
