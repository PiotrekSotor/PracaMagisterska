\chapter{Charakterystyka algorytmów ewolucyjnych}
%Nocedal and Wright (1999) is a comprehensive reference for the previous three methods. z dokumentacji R
%\begin{itemize}
%\item opierają się na okreslonym zbiorze osobników tworzacych populacje
%\item powstały przez analogie do obserwowanych w naturze zachowań
%\item sterowane przez parametry odpowiadające za różne aspekty
%\item dlaczego wybrano akurat takie algorytmy do porownania (to że genetyczny wykonuje się szybciej, bo mniej obliczeń, ale znajdzie rozwiązanie w większej liczbie iteracji, PSO dla porównania)
%\item czym różnią się tutaj przedstawione algorytmy od strategi ewolucyjnych - niezmienność parametrów w czasie działania algorytmów
%\item zaznaczenie że testy oparto o implementacje algorytmów zawarte w pakiecie GA dla języka R
%\end{itemize}
\par
Ludzie od zawsze czerpali z rozwiązań wypracowanych przez tysiące lat w naturze dzięki ewolucji. W rzeczywistym świecie konieczność przystosowania się do zmieniających się warunków życia determinuje ciągłe adaptowanie się populacji gatunków. Większość rozwiązań może okazać się niewystarczająco efektywna. Część lub nawet tylko jedno z nich z pewnością spełni jednak wymagania środowiska - przeciwnym razie gatunek czeka wymarcie. 
\par
Takie właśnie podejście zdecydowano się wykorzystać podchodząc do koncepcji znajdowania optymalnych (lub jemu bliskich) rozwiązań problemów różnego typu. Charakterystyczną cechą ogółu algorytmów ewolucyjnych jest oparcie swojego działania o jednoczesne rozpatrywanie wielu możliwych rozwiązań i manipulowaniu nimi w celu polepszenia ich jakości. Definicję jakości określa się dla konkretnego rodzaju problemu. Przez analogię do natury przyjmuje się, że osobniki najlepiej przystosowane, a więc o rozwiązaniu bliższym szukanego optimum, silniej oddziałują na populację niż te gorsze. Kolejne iteracje algorytmów genetycznych i memetycznych należy postrzegać jako następujące po sobie pokolenia osobników, zaś w przypadku optymalizacji rojem cząstek wykonania głównej pętli odpowiadają pojedynczym jednostkom zdyskretyzowanego czasu symulacji zachowania stada. 
\par
Celem pracy jest zbadanie efektywności algorytmów memetycznych i badanie to zostanie przeprowadzone poprzez porównanie skuteczności jego działa do algorytmów genetycznych i PSO. Dokładne kryteria porównawcze oraz przyjęte miary efektywności zostaną opisane w dalszej części pracy. Praca każdego z algorytmów może być sterowana poprzez charakterystyczną dla niego grupę parametrów określających działanie kluczowych dla algorytmu operacji. 
\par 
Rozwinięciem koncepcji algorytmów ewolucyjnych jest pojęcie \emph{strategi ewolucyjnych}. W ogólnym rozumieniu nie różnią się w znacznym stopniu od nich, jednakże wprowadzają dodatkowe mechanizmy umożliwiające poprawę jakości znajdowanych rezultatów lub szybkość ich otrzymywania. Najprostszym modelem strategi ewolucyjnej jest stopniowe zmienianie parametrów algorytmu ewolucyjnego wraz z postępem obliczeń i zachowaniem populacji. Jednym z problemów algorytmów ewolucyjnych jest obserwowana przedwczesna zbieżność populacji wokół nieoptymalnego rozwiązania. Odpowiednio przygotowana strategia ewolucyjna może taką sytuację wykryć i odpowiednio do niej zaadaptować parametry dalszej pracy. Strategie ewolucyjne nie są jednak przedmiotem tego wywodu.
\par
W zależności od przyjętej koncepcji założenia działania omawianych algorytmów mogą się nieznacznie różnić. Ta praca została oparta na implementacji algorytmu memetycznego zaproponowanej przez Lucca Scruca w pakiecie dla języka \emph{R ''GA''}\cite{gaPackage}. Podobnie w przypadku algorytmu PSO została wykorzystana implementacja zawarta w pakiecie \emph{''PSO''} autorstwa Clausa Bendtsena\cite{psoPackage}. W przypadku wszystkich opisanych w tym rozdziale algorytmów za \emph{warunkiem stopu} można rozumieć wykonanie z góry określonej maksymalnej liczby iteracji pętli lub brak poprawy najlepszego rozwiązania w pewnej liczbie ostatnich przebiegów algorytmu. 

\section{Algorytm genetyczny}
\label{sec:algorytm_genetyczny}
%\begin{itemize}
%\item rys historyczny, kiedy pierwszy raz wykorzystywane
%\item jak rozumieć materiał genetyczny w kontekście optymalizacji problemu
%\item opierają się na populacji zmieniającej się wraz z następowaniem kolejnych pokoleń osobników
%\item wykorzystuje naturalne mechanizmy tworzenia potomstwa, mutowania osobników
%\item moze być sterowany zarówno przez okreslanie prawdopodobieństw wystąpienia mechanizmów opisanych jak i samą ich mechanikę
%\item mechanizmy muszą dbać o to by osobniki nadal były zgodne z dziedziną problemu
%\item pseudokod
%\end{itemize}
\par 
Prace nad implementacją algorytmu stworzonego na podobieństwo zachowań pomiędzy osobnikami w populacji zaczęły się już w latach 50' \cite{barker1958simulation}. Początkowo była to próba symulowania w środowisku komputerowym ewolucji biologicznej, ale już w latach 60' dostrzeżono potencjał wykorzystania tej metody do rozwiązywania problemów optymalizacyjnych \cite{bremermann1962optimization}.
\par
Zastosowanie mechanizmu ewolucji populacji w celu rozwiązywania rzeczywistych problemów wymaga zdefiniowania dwóch podstawowych rzeczy: reprezentacji i funkcji dopasowania (przystosowania) osobników. W przypadku rozwiązywania problemów rzeczywistoliczbowych osobniki mogą kodować rozwiązanie poprzez konkretne wartości optymalizowanych parametrów. Reprezentacje i sposób interpretacji genów wybiera na podstawie rodzaju rozpatrywanego problemu - inne reprezentacje stosuje się w problemach kombinatorycznych i permutacyjnych. Funkcja dopasowania jest miarą, dzięki której możliwe jest wzajemne porównanie dwóch osobników ze sobą. Determinuje również kryterium optymalizacji danego problemu. Jednoznacznie określa wartość danego osobnika w kontekście poszukiwania optymalnego rozwiązania. 
\par
Podstawową koncepcją na której opiera się działanie i skuteczność algorytmów genetycznych jest zaaplikowanie znanego ze świata rzeczywistego procesu zastępowania starego pokolenia młodszymi osobnikami. Przez to jednym z parametrów specyfikujących działanie algorytmu jest liczba symulowanych pokoleń, w trakcie których odbywają się mechanizmy ewolucji. Podstawowymi operacjami wykonywanymi na populacji są selekcja, mutacja i krzyżowanie osobników \cite{sudholt2008computational}. Podstawową formę algorytmu genetycznego można przedstawić w sposób przedstawiony na listingu \ref{lst:genetic_algorithm_psedocode}. Za populację początkową można przyjąć losowe wygenerowane osobniki zgodne z dziedziną i założeniami rozwiązywanego problemu.
\begin{lstlisting}[caption=Podstawowy schemat algorytmu genetycznego, label=lst:genetic_algorithm_psedocode, mathescape]
Inicjuj populację $\mu$ osobników
$\textbf{powtarzaj}$
    Przeprowadź selekcję osobników z obecnej populacji
    Na podstawie wybranych stwórz $\lambda$ potomków poprzez krzyżowanie
    Na wybranych osobnikach wykonaj mutacje
    Z otrzymanej puli wybierz $\mu$ osobników
    Zastąp obecną populację nową
$\textbf{dopóki}$ warunek stopu
\end{lstlisting}
\par
Podstawowy model bywa rozszerzany o procedurę chronienia jednego lub więcej osobników o najlepszym przystosowaniu z starej populacji. Badania dowodzą, że takie podejście znacznie poprawia stabilność i powtarzalność otrzymywanych rezultatów algorytmu \cite{baluja1995removing}. Tak również postąpiono w trakcie przeprowadzania badań w ramach tej pracy - wykorzystano domyślny w pakiecie \emph{GA} rozmiar grupy chronionych osobników wynoszący 5\% populacji.
\par
Selekcji osobników z obecnej populacji do późniejszej reprodukcji osobników zwykle dokonuje się wykorzystując wartość funkcji przystosowania. Najczęściej stosowane do tego są metody \emph{rankingowa} i \emph{ruletki}. Pierwsza z nich szereguje osobniki według ich przystosowania, a następnie wybiera żądaną ich liczbę najlepszych. W dalszym przetwarzaniu wszystkie wybrane osobniki mają równe szanse na wzięcie udziału w reprodukcji. Utracona zostaje więc różnica w jakości pomiędzy najlepszymi i najgorszymi z wyselekcjonowanej puli osobników. Alternatywą dla takiego podejścia jest druga metoda. Polega ona na tym, że z pełnej puli populacji losuje się ze zwracaniem osobniki, przy czym prawdopodobieństwo wylosowania danego osobnika jest proporcjonalne do jego jakości \cite{sudholt2008computational}. Obie metody można jednak połączyć otrzymując algorytm polegający na wstępnej selekcji podzbioru populacji na którym następnie dokonuje się losowań. Przeprowadzając badania posłużono się metodą \emph{ruletkową}.
\par
O ile wyżej opisane metody selekcji osobników swobodnie można stosować niezależnie od specyfiki problemu i reprezentacji rozwiązania, o tyle działanie operatorów krzyżowania i mutacji musi zostać dopasowane do konkretnego zastosowania. Sposób kodowania rozwiązania w osobnikach wpływa jednoznacznie na możliwe operowanie nimi, a specyfika zastosowania wyznacza warunki jakie muszą spełniać nowe osobniki, by mogły być traktowane jako prawidłowe rozwiązanie (prawidłowe w kontekście dziedziny problemu, nie jego jakości rozwiązania). Dokładny opis wykorzystanych operatorów mutacji i krzyżowania zostanie przedstawiony w rozdziale \ref{ch:przyjety_model_doboru_parametrow_algorytmu}.
\par

\section{Algorytm memetyczny}
%\begin{itemize}
%\item rys historyczny, czym jest mem
%\item wspomnienie o Lamarkianizmie i tym drugim
%\item o tym że jest to rozwinięcie koncepcji algorytmu genetycznego
%\item możliwe warianty memetycznego realizowane w literaturze, różne miejsca wykonywania lokalnego przeszukania
%\item znaczenie parametrów algorytmu
%\item pseudokod czy coś takiego
%
%\end{itemize}
\par
\emph{Mem} jest pojęciem wprowadzonym do użycia w 1976 w książce Richarda Dawkinsa \emph{Samolubny gen} \cite{dawkins2011}. Użył go do nazwania podstawowej jednostki w ewolucji kulturowej, czegoś na podobieństwo genu w ewolucji genetycznej. Opiera się to na obserwacji zachowania populacji, w której spotykające się osobniki wymieniają się ze sobą informacjami, przejmują dobre cechy od sąsiednich przedstawicieli lub same poprawiają się w własnym zakresie. Podstawową różnicą pomiędzy \emph{memem}, a \emph{genem} jest fakt, że \emph{mem} jest bardziej związany z fenotypem osobnika, niż z jego genetycznym niezmiennym stanem. Fenotyp w prawdzie jest w znacznej części determinowany przez genotyp, jednakże dodatkowo warunkowany jest przez środowisko zewnętrzne. W kontekście ewolucji memetycznej ugruntowały się dwa podejścia definiujące trwałość i dziedziczność cech będących skutkiem oddziaływania środowiska na osobniki: Baldwinizm i Lamarkizm. Pierwsze z nich zakłada, że materiał genetyczny nie może ulec zmianie w trakcie życia osobnika, zatem nabyte cechy nie są przekazywane potomstwu w genetyczny sposób. Informacje o tym, jak jednostka udoskonaliła się w trakcie swojego pokolenia może zostać jednak przekazana w innej postaci, co determinuje osobny sposób kodowania cech przekazywanych \emph{genetycznie} i \emph{memetycznie}. W przeciwnym razie lepsze przystosowanie osobnika będzie widoczne tylko w postaci jego fenotypu i nie zostanie przekazane potomstwu. Drugie podejście wbrew temu co mówi rzeczywista biologia przyjmuje, że ulepszenie jednostki jest trwałe - zapisane w materiale genetycznym propaguje się na potomstwo \cite{whitley1994lamarckian}.
\par
\emph{Algorytm memetyczny} najczęściej realizuje podejście zgodne z Lamarkizmem. Swoją budowę i zasadę działania opiera na \emph{algorytmie genetycznym} dodając do niego charakterystyczne dla siebie mechanizmy. Opisane powyżej zachowanie osobników przyjmujących dobre cechy napotkanych sąsiadów realizuje się poprzez zastosowanie algorytmów przeszukiwania lokalnego na konkretnych, zwykle losowo wybranych jednostkach. W literaturze można znaleźć przykłady implementujące doskonalenie osobników na różnych etapach działania algorytmu genetycznego. Najczęściej stosowanym jest model aplikujący przeszukiwanie lokalne przed selekcją osobników do reprodukcji \cite{maringer2006portfolio} lub po wytworzeniu potomków drogą krzyżowania a przed selekcją osobników tworzących nowe pokolenie \cite{sudholt2008computational}. Niekiedy można również stosować przeszukiwanie lokalne na etapie inicjacji populacji. Każdy z losowo wygenerowanych osobników poddaje się lokalnej optymalizacji \cite{elbeltagi2005comparison}.
\par
Przyjęty w czasie badań model algorytmu memetycznego przedstawiono na listingu \ref{lst:memetic_algorithm_pseudocode}. Pogrubieniem zaznaczono element odróżniający procedurę od wcześniej przestawionego na listingu \ref{lst:genetic_algorithm_psedocode} dla schematu działania algorytmu genetycznego. Jest to model zrealizowany przez autora pakieru \emph{GA} dla języka \emph{R}. Na chwilę obecną pakiet ten umożliwia optymalizację przy wykorzystaniu algorytmu memetycznego jedynie problemów rzeczywistoliczbowych. Oferuje ponad to możliwość wyboru spośród 6 operatorów przeszukania lokalnego. Jeden z nich - algorytm Brenta - stosowany może być jedynie w przypadku optymalizacji problemów jednowymiarowych, przez co nie będzie brany pod uwagę w porównaniach. Dokładny opis pozostałych operatorów znajduje się w rodziale \ref{ch:przyjety_model_doboru_parametrow_algorytmu}.  
\begin{lstlisting}[caption=Model algorytmu memetycznego stosowany w pakiecie \emph{GA}, label=lst:memetic_algorithm_pseudocode, mathescape]
Inicjuj populację $\mu$ osobników
$\textbf{powtarzaj}$
    $\textbf{Na rankingowo wybranym osobniku przeprowadź przeszukiwanie lokalne}$
    Przeprowadź selekcję osobników z obecnej populacji
    Na podstawie wybranych stwórz $\lambda$ potomków poprzez krzyżowanie
    Na wybranych osobnikach wykonaj mutacje
    Z otrzymanej puli wybierz $\mu$ osobników
    Zastąp obecną populację nową
$\textbf{dopóki}$ warunek stopu
$\textbf{Na najlepszym osobniku przeprowadź przeszukiwanie lokalne}$
\end{lstlisting}
\par
W tym momencie należy zauważyć specyfikę takiego, a nie innego przyjętego modelu przez autora pakietu \emph{GA}. Doskonaleniu w czasie trwania populacji może zostać poddany jedynie jeden osobnik z populacji, gdzie w przypadku innych opisywanych publikacji autorzy wykonywali tę operację nawet na całej populacji \cite{elbeltagi2005comparison}\cite{sudholt2008computational}\cite{ong2006classification}.

\section{Algorytm optymalizacji rojem cząstek - PSO}
\label{sec:pso}
\par
Pierwsze założenia działania algorytmu optymalizacji rojem cząstek opisali Kennedy i Eberhart w 1995 roku \cite{kennedy1995pso}. Początkowo mięli oni na celu utworzenie sztucznej inteligencji mogącej imitować interakcje społeczne \cite{poli2007particle}. Oparli swoje rozwiązanie na obserwacjach zachowania ptaków stadnie poszukujących pożywienia. Przyjmując ilość pożywienia w danym miejscu za funkcję przystosowania położenia ptaka można rozumieć problem znalezienia optymalnego miejsca do żeru stada jako klasyczny problem znalezienia ekstremum globalnego w przestrzeni trójwymiarowej.
\par
Odpowiednikiem rzeczywistego stada ptaków jest tytułowy \emph{rój cząstek}. Każda z nich opisywana jest przy pomocy trzech \emph{d}-wymiarowych wektorów opisujących obecne położenie cząstki ($\vec{x_i}$), aktualną prędkość ($\vec{v_i}$) oraz najlepszą dotychczasową pozycję ($\vec{p_i}$). Najlepsza uzyskana wartość funkcji dopasowania w pozycji $\vec{p_i}$ oznacza się jako $pbest_i$. Rezultatem działania algorytmu jest najlepsza znaleziona w ten sposób pozycja $\vec{p_i}$ wraz z odpowiadającą jej $pbest_i$.
\par
Jednym z wariantów PSO jest określenie parametru \emph{wagi inercji}. Oznacza to skłonność cząstek do ulegania wpływom otaczających ją cząstek lub jej \emph{uparte} podążanie w wybranym przez siebie kierunku. Za wspomniany parametr odpowiada \emph{w} w wzorze \ref{eq:pso_v}. Wysoka wartość wagi inercji zwiększa eksploracyjną zdolność populacji - powiększana jest przeszukana przestrzeń rozwiązań. Z kolei niska wartość parametru przyspiesza zbieżność działania algorytmu. W zależności od rozwiązywanego problemu różne zachowania roju mogą uzyskiwać zadowalające rezultaty.
\begin{lstlisting}[caption=Schemat algorytmu optymalizacji rojem cząstek - \textbf{PSO}, label=lst:pso_pseudocode, mathescape, breaklines=true]
Inicjuj populację $\mu$ osobników, ich położenie i prędkość
$\textbf{powtarzaj}$
    Dla każdego osobnika wyznacz wartość funkcji przystosowania 
        przystosowania 
    Wyznacz nowe wartości $pbest_i$ i $\vec{p_i}$
    Dla każdego osobnika wyznacz najlepiej przystosowanego osobnika 
        z otoczenia i przypisz do $\vec{p_g}$
    Wyznacz nowe prędkości i położenia osobników populacji
        ze wzorów $\ref{eq:pso_v}$ i $\ref{eq:pso_x}$
$\textbf{dopóki}$ warunek stopu
\end{lstlisting}

\begin{equation} \label{eq:pso_v}
\vec{v_i} \leftarrow \emph{w}\vec{v_i} + \vec{U}(0,\phi_1)\otimes(\vec{p_i}-\vec{x_i}+\vec{U}(0,\phi_2)\otimes(\vec{p_g}-\vec{x_i})
\end{equation}
\begin{equation} \label{eq:pso_x}
\vec{x_i} \leftarrow \vec{x_i} + \vec{v_i}
\end{equation}
W wzorach \ref{eq:pso_v} i \ref{eq:pso_x} zastosowano następujące oznaczenia:
\begin{itemize}
\item $\vec{U}(0,\phi_i)$ przedstawia wektor wartości wylosowanych z rozkładem jednostajnym z przedziału $[0, \phi_i]$, losowanie powtarzane jest dla każdej cząstki przy każdej iteracji,
\item $\otimes$ oznacza mnożenie elementów dwóch wektorów przez siebie (\emph{eng. component-wise multiplication},
\item $\phi_1$ oraz $\phi_2$ opisują \emph{współczynniki przyspieszenia} w kierunku lokalnego i populacyjnego najlepszego znanego rozwiązania,
\item $w$ oznacza \emph{wagę inercji}
\end{itemize}

\section{Specyfika optymalizacji problemów rzeczywistoliczbowych}
\label{sec:specyfika_optymalizacji_problemów_rzeczywistoliczbowych}
%\begin{itemize}
%\item konieczność spełniania constraintów na parametrach
%\item niejednorodne znaczenie i wpływ poszczególnych parametrów na wartości funkcji celu
%\item potencjalnie nieskończona przestrzeń możliwych rozwiązań
%\end{itemize}
\par
Dziedzina problemów rzeczywistoliczbowych jest bardzo szeroka. Jednak same metody optymalizacji tych zagadnień są w większości przenaszalne bez względu na wymiarowość zagadnienia. Jedyne czego wymagają wyżej opisane algorytmy to jawnie zdefiniowana funkcja pozwalająca w jednoznaczny i powtarzalny sposób określić jakość danego rozwiązania. Wyróżnia to w znacznym stopniu ten rodzaj problemów od \emph{kombinatorycznych} i \emph{permutacyjnych} - te wymagają najczęściej również odpowiedniego dopasowania użytych w ich przypadkach operatorów mutacji lub krzyżowania.
\par
Liczby rzeczywiste wykorzystywane do określania wartości parametrów optymalizowanej funkcji wymagają jawnego określenia obszaru przeszukiwań - dziedziny każdego z parametrów. Owa dziedzina może przybierać mniej lub bardziej złożone warunki. Podstawowym ograniczeniem nakładanym na argumenty funkcji jest przedział $[minimum \quad maksimum]$. Zdefiniowanie takiego przedziału choć zawęża krąg poszukiwań, nie zmienia faktu teoretycznego istnienia nieskończonej liczby możliwych wartości do przeszukania. Na szczęście \emph{teoretyczna} nieskończona liczba jest z góry ograniczona przez skończoną dokładność reprezentacji liczb w systemie komputerowym. Oprócz określenia górnych i dolnych krańców przestrzeni rozwiązań możliwe jest zdefiniowanie dodatkowych obostrzeń dotyczących zarówno wartości jednego parametru jak i wzajemnych relacji pomiędzy nimi. Wszystko to musi zostać wyspecyfikowane zgodnie z konkretnym rozwiązywanym problemem.
\par
Wraz z dużą liczbą wymiarów analizowanego zagadnienia narasta problem związany z niejednorodnym wpływem każdego z parametrów osobnika na wartość funkcji jego dostosowania. Może zdarzyć się tak, że każda niewielka zmiana wartości jednego z atrybutów jednostki diametralnie zmieni jej jakość, natomiast skrajna zmiana innego nie wpłynie znacząco na jej ocenę. Z reguły powyżej opisane algorytmu powinny radzić sobie z takimi sytuacjami, jednakże z pewnością pomocne w takich przypadkach będą dodatkowe metody wstępnej analizy funkcji dopasowania pozwalające na redukcję wymiarowości problemu lub maksymalne zawężenie przyjętych zakresów wartości atrybutów. Wykorzystane w tej pracy funkcje testowe nie będą wymagały definiowania złożonych warunków odnośnie wartości parametrów. 


\section{Przykłady zastosowań}
%\begin{itemize}
%\item powszechnie stosowana w wielu różnych zastosowaniach
%\item stosowana w metodach złożonych modeli matematycznych, których analiza teoretyczna jest bardzo trudna
%\item stosowana w przypadkach braku modeli matematycznych, operając się na estymacjach i znanych pomiarach
%\item w praktyce zastępuje metodę prób i błędów
%\item wybrać parę problemów z pdfa dla cec2011 \cite{cec2011}
%\end{itemize}
\par
Wspomniana w podrozdziale \ref{sec:specyfika_optymalizacji_problemów_rzeczywistoliczbowych} wysoka przenaszalność oraz elastyczność odnośnie zarówno wymiarowości optymalizowanego rozwiązania pozwala na szerokie stosowanie wyżej opisanych metod. Są to metody szczególnie pomocne badaczom w przypadkach rozpatrywania problemów, których skomplikowanie i złożoność modelu matematycznego nie pozwala na znalezienie optymalnego rozwiązania drogą analityczną. Co więcej - nic nie stoi na przeszkodzie takiemu zdefiniowaniu funkcji dopasowania, by jej wartość zależała nie od zadeklarowanego jawnie wzoru matematycznego lecz była określana na podstawie zbioru rzeczywistych pomiarów. 
\par
Zastosowanie algorytmicznej optymalizacji pozwala zespołom inżynierskim na szybkie sprawdzanie i doskonalenie koncepcji przez nich rozwiązywanych. Dzięki dokładnemu matematycznemu opisowi analizowanego problemu grupa jest w stanie zdecydowanie zmniejszyć liczbą produkcji kosztownych prototypów testowych dzięki temu, że w wyniku wykorzystania metod optymalizacji wyselekcjonowane zostaną tylko te rozwiązania, których skuteczność jest najbardziej prawdopodobna. Innymi słowy - w pewnym stopniu wykonuje za inżynierów znajdowanie rozwiązania metodą prób i błędów. 
\par
Dobrym źródłem przykładów problemów rzeczywistoliczbowych rozwiązywanych poprzez optymalizację są materiały związane z różnymi konkursami algorytmicznymi. Pierwsze z poniżej omówionych zagadnień opisano w ramach \emph{CEC 2011} \cite{cec2011}, zaś drugie pochodzi z publikacji dotyczącej procesu projektowania samolotu o konstrukcji latającego skrzydła \cite{lee2007multi}.
\par
\textbf{Problem I:} \emph{estymacja parametrów fali dźwiękowej modulowanej częstotliwościowo} jest zagadnieniem ważnym w nowoczesnych systemach muzycznych. Zadanie sprowadza się do znalezienia wektora 6 parametrów $X = \lbrace\alpha_1, \omega_1, \alpha_2, \omega_2, \alpha_3, \omega_3\rbrace$ dla których minimalny będzie błąd kwadratowy fali uzyskanej wzorem \ref{eq:problem_I_funkcja_przystosowania}, a przebiegiem fali rzeczywistej. Na potrzeby laboratoryjnej realizacji optymalizacji za rzeczywisty dźwięk przyjęto falę opisaną wzorem \ref{eq:problem_I_fala_odniesienia}. Przyjęto, że $\theta = 2\pi/100$, a parametry zawierają się w przedziale~$[-6.4\quad6.35]$.
\begin{equation} \label{eq:problem_I_funkcja_przystosowania}
y(t)=\alpha_1*\sin(\omega_1 t\theta + \alpha_2*\sin(\omega_2 t\theta + \alpha_3*\sin(\omega_3 t\theta)))
\end{equation}
\begin{equation} \label{eq:problem_I_fala_odniesienia}
y(t)=(1.0)*\sin((5.0) t\theta + (-1.5)*\sin((4.8) t\theta + (2.0)*\sin((4.9) t\theta)))
\end{equation}
\par
\textbf{Problem II:} \emph{optymalizacja kształtu skrzydeł} konstrukcji samolotu o złączonych skrzydłach. Celem inżynierów była poprawa charakterystyk aerodynamicznych przy jednoczesnym zmniejszeniu ich skutecznej powierzchni \cite{lee2007multi}. Jest to jedno z kluczowych zagadnień związanych z projektowaniem trudno wykrywalnych samolotów. Koncepcja samolotów o złączonych skrzydłach jest rozwijana już od 1919 roku, a najbardziej rozpoznawalnym przykładem takiej konstrukcji jest amerykański bombowiec B-2 Spirit. 
\par
