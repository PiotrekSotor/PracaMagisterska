\chapter{Charakterystyka algorytmów ewolucyjnych}
%\begin{itemize}
%\item opierają się na okreslonym zbiorze osobników tworzacych populacje
%\item powstały przez analogie do obserwowanych w naturze zachowań
%\item sterowane przez parametry odpowiadające za różne aspekty
%\item dlaczego wybrano akurat takie algorytmy do porownania (to że genetyczny wykonuje się szybciej, bo mniej obliczeń, ale znajdzie rozwiązanie w większej liczbie iteracji, PSO dla porównania)
%\item czym różnią się tutaj przedstawione algorytmy od strategi ewolucyjnych - niezmienność parametrów w czasie działania algorytmów
%\item zaznaczenie że testy oparto o implementacje algorytmów zawarte w pakiecie GA dla języka R
%\end{itemize}
\par
Ludzie od zawsze czerpali z rozwiązań wypracowanych przez tysiące lat w naturze dzięki ewolucji. W rzeczywistym świecie konieczność przystosowania się do zmieniających się warunków życia determinuje ciągłe adaptowanie się populacji gatunków. Większość rozwiązań może okazać się niewystarczająco efektywna. Część lub nawet tylko jedno z nich z pewnością spełni jednak wymagania środowiska - przeciwnym razie gatunek czeka wymarcie. 
\par
Takie właśnie podejście zdecydowano się wykorzystać podchodząc do koncepcji znajdowania optymalnych (lub jemu bliskich) rozwiązań problemów różnego typu. Charakterystyczną cechą ogółu algorytmów ewolucyjnych jest oparcie swojego działania o jednoczesne rozpatrywanie wielu możliwych rozwiązań i manipulowaniu nimi w celu polepszenia ich jakości. Definicję jakości określa się dla konkretnego rodzaju problemu. Przez analogię do natury przyjmuje się, że osobniki najlepiej przystosowane, a więc o rozwiązaniu bliższym szukanego optimum, silniej oddziaływują na populację niż te gorsze. 
\par
Celem pracy jest zbadanie efektywności algorytmów memetycznych i badanie to zostanie przeprowadzone poprzez porównanie skuteczności jego działa do algorytmów genetycznych i PSO. Dokładne kryteria porównawcze oraz przyjęte miary efektywności zostaną opisane w dalszej części pracy. Praca każdego z algorytmów może być sterowana poprzez charakterystyczną dla niego grupę parametrów określających działanie kluczowych dla algorytmu operacji. 
\par 
Rozwinięciem koncepcji algorytmów ewolucyjnych jest pojęcie \emph{strategi ewolucyjnych}. W ogólnym rozumieniu nie różnią się w znacznym stopniu od nich, jednakże wprowadzają dodatkowe mechanizmy umożliwiające poprawę jakości znajdowanych rezultatów lub szybkość ich otrzymywania. Najprostszym modelem strategi ewolucyjnej jest stopniowe zmienianie parametrów algorytmu ewolucyjnego wraz z postępem obliczeń i zachowaniem populacji. Jednym z problemów algorytmów ewolucyjnych jest obserwowana przedwczesna zbieżność populacji wokół nieoptymalnego rozwiązania. Odpowiednio przygotowana strategia ewolucyjna może taką sytuację wykryć i odpowiednio do niej zaadaptować parametry dalszej pracy. Strategie ewolucyjne nie są jednak przedmiotem tego wywodu.
\par
W zależności od przyjętej koncepcji założenia działania omawianych algorytmów mogą się nieznacznie różnić. Ta praca została oparta na implementacji algorytmu memetycznego zaproponowanej przez Lucca Scruca w pakiecie dla języka \emph{R ''GA''}\cite{gaPackage}. Podobnie w przypadku algorytmu PSO została wykorzystana implementacja zawarta w pakiecie \emph{''PSO''} autorstwa Clausa Bendtsena\cite{psoPackage}.


\section{Algorytm genetyczny}
\begin{itemize}
\item rys historyczny, kiedy pierwszy raz wykorzystywane
\item jak rozumieć materiał genetyczny w kontekście optymalizacji problemu
\item opierają się na populacji zmieniającej się wraz z następowaniem kolejnych pokoleń osobników
\item wykorzystuje naturalne mechanizmy tworzenia potomstwa, mutowania osobników
\item moze być sterowany zarówno przez okreslanie prawdopodobieństw wystąpienia mechanizmów opisanych jak i samą ich mechanikę
\item mechanizmy muszą dbać o to by osobniki nadal były zgodne z dziedziną problemu
\item pseudokod
\item opis krosowania i mutacji wybranego do testowania
\end{itemize} 
Prace nad implementacją algorytmu stworzonego na podobieństwo zachowań pomiędzy osobnikami w populacji zaczęły się już w latach 50' \cite{barker1958simulation}. 

\section{Algorytm memetyczny}
\begin{itemize}
\item rys historyczny, czym jest mem
\item wspomnienie o Lamarkianizmie i tym drugim
\item o tym że jest to rozwinięcie koncepcji algorytmu genetycznego
\item możliwe warianty memetycznego realizowane w literaturze, różne miejsca wykonywania lokalnego przeszukania
\item znaczenie parametrów algorytmu
\item pseudokod czy coś takiego
\item opis testowanych operatorów lokalnego przeszukiwania
\end{itemize}

\section{Algorytm optymalizacji rojem cząstek - PSO}
\begin{itemize}
\item rys historyczny
\item symuluje zachowanie stada, roju owadów
\item pseudokod
\item opis przyjętych parametrów
\end{itemize}

\section{Specyfika optymalizacji problemów rzeczywistoliczbowych}
\begin{itemize}
\item powszechnie stosowana w wielu różnych zastosowaniach
\item konieczność spełniania constraintów na parametrach
\item niejednorodne znaczenie i wpływ poszczególnych parametrów na wartości funkcji celu
\end{itemize}

\subsection{Przykłady zastosowań}