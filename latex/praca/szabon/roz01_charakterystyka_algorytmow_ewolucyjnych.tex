\chapter{Charakterystyka algorytmów ewolucyjnych}
%\begin{itemize}
%\item opierają się na okreslonym zbiorze osobników tworzacych populacje
%\item powstały przez analogie do obserwowanych w naturze zachowań
%\item sterowane przez parametry odpowiadające za różne aspekty
%\item dlaczego wybrano akurat takie algorytmy do porownania (to że genetyczny wykonuje się szybciej, bo mniej obliczeń, ale znajdzie rozwiązanie w większej liczbie iteracji, PSO dla porównania)
%\item czym różnią się tutaj przedstawione algorytmy od strategi ewolucyjnych - niezmienność parametrów w czasie działania algorytmów
%\item zaznaczenie że testy oparto o implementacje algorytmów zawarte w pakiecie GA dla języka R
%\end{itemize}
\par
Ludzie od zawsze czerpali z rozwiązań wypracowanych przez tysiące lat w naturze dzięki ewolucji. W rzeczywistym świecie konieczność przystosowania się do zmieniających się warunków życia determinuje ciągłe adaptowanie się populacji gatunków. Większość rozwiązań może okazać się niewystarczająco efektywna. Część lub nawet tylko jedno z nich z pewnością spełni jednak wymagania środowiska - przeciwnym razie gatunek czeka wymarcie. 
\par
Takie właśnie podejście zdecydowano się wykorzystać podchodząc do koncepcji znajdowania optymalnych (lub jemu bliskich) rozwiązań problemów różnego typu. Charakterystyczną cechą ogółu algorytmów ewolucyjnych jest oparcie swojego działania o jednoczesne rozpatrywanie wielu możliwych rozwiązań i manipulowaniu nimi w celu polepszenia ich jakości. Definicję jakości określa się dla konkretnego rodzaju problemu. Przez analogię do natury przyjmuje się, że osobniki najlepiej przystosowane, a więc o rozwiązaniu bliższym szukanego optimum, silniej oddziaływują na populację niż te gorsze. 
\par
Celem pracy jest zbadanie efektywności algorytmów memetycznych i badanie to zostanie przeprowadzone poprzez porównanie skuteczności jego działa do algorytmów genetycznych i PSO. Dokładne kryteria porównawcze oraz przyjęte miary efektywności zostaną opisane w dalszej części pracy. Praca każdego z algorytmów może być sterowana poprzez charakterystyczną dla niego grupę parametrów określających działanie kluczowych dla algorytmu operacji. 
\par 
Rozwinięciem koncepcji algorytmów ewolucyjnych jest pojęcie \emph{strategi ewolucyjnych}. W ogólnym rozumieniu nie różnią się w znacznym stopniu od nich, jednakże wprowadzają dodatkowe mechanizmy umożliwiające poprawę jakości znajdowanych rezultatów lub szybkość ich otrzymywania. Najprostszym modelem strategi ewolucyjnej jest stopniowe zmienianie parametrów algorytmu ewolucyjnego wraz z postępem obliczeń i zachowaniem populacji. Jednym z problemów algorytmów ewolucyjnych jest obserwowana przedwczesna zbieżność populacji wokół nieoptymalnego rozwiązania. Odpowiednio przygotowana strategia ewolucyjna może taką sytuację wykryć i odpowiednio do niej zaadaptować parametry dalszej pracy. Strategie ewolucyjne nie są jednak przedmiotem tego wywodu.
\par
W zależności od przyjętej koncepcji założenia działania omawianych algorytmów mogą się nieznacznie różnić. Ta praca została oparta na implementacji algorytmu memetycznego zaproponowanej przez Lucca Scruca w pakiecie dla języka \emph{R ''GA''}\cite{gaPackage}. Podobnie w przypadku algorytmu PSO została wykorzystana implementacja zawarta w pakiecie \emph{''PSO''} autorstwa Clausa Bendtsena\cite{psoPackage}.


\section{Algorytm genetyczny}
%\begin{itemize}
%\item rys historyczny, kiedy pierwszy raz wykorzystywane
%\item jak rozumieć materiał genetyczny w kontekście optymalizacji problemu
%\item opierają się na populacji zmieniającej się wraz z następowaniem kolejnych pokoleń osobników
%\item wykorzystuje naturalne mechanizmy tworzenia potomstwa, mutowania osobników
%\item moze być sterowany zarówno przez okreslanie prawdopodobieństw wystąpienia mechanizmów opisanych jak i samą ich mechanikę
%\item mechanizmy muszą dbać o to by osobniki nadal były zgodne z dziedziną problemu
%\item pseudokod
%\end{itemize}
\par 
Prace nad implementacją algorytmu stworzonego na podobieństwo zachowań pomiędzy osobnikami w populacji zaczęły się już w latach 50' \cite{barker1958simulation}. Początkowo była to próba symulowania w środowisku komputerowym ewolucji biologicznej, ale już w latach 60' dostrzeżono potencjał wykorzystania tej metody do rozwiązywania problemów optymalizacyjnych \cite{bremermann1962optimization}.
\par
Zastosowanie mechanizmu ewolucji populacji w celu rozwiązywania rzeczywistych problemów wymaga zdefiniowania dwóch podstawowych rzeczy: reprezentacji i funkcji dopasowania osobników. W przypadku rozwiązywania problemów rzeczywistoliczbowych osobniki mogą kodować rozwiązanie poprzez konkretne wartości optymalizowanych parametrów. Reprezentacje i sposób interpretacji genów wybiera na podstawie rodzaju rozpatrywanego problemu - inne reprezentacje stosuje się w problemach kombinatorycznych i permutacyjnych. Funkcja dopasowania jest miarą, dzięki której możliwe jest wzajemne porównanie dwóch osobników ze sobą. Determinuje również kryterium optymalizacji danego problemu. Jednoznacznie określa wartość danego osobnika w kontekście poszukiwania optymalnego rozwiązania. 
\par
Podstawową koncepcją na której opiera się działanie i skuteczność algorytmów genetycznych jest zaaplikowanie znanego ze świata rzeczywistego procesu zastępowania starego pokolenia młodszymi osobnikami. Przez to jednym z parametrów specyfikujących działanie algorytmu jest liczba symulowanych pokoleń, w trakcie których odbywają się mechanizmy ewolucji. Podstawowymi operacjami wykonywanymi na populacji są selekcja, mutacja i krzyżowanie osobników \cite{sudholt2008computational}. Podstawową formę algorytmu genetycznego można przedstawić w sposób przedstawiony na listingu \ref{lst:genetic_algorithm_psedocode}. Za populację początkową można przyjąć losowe wygenerowane osobniki zgodne z dziedziną i założeniami rozwiązywanego problemu.
\begin{lstlisting}[caption=Podstawowy schemat algorytmu genetycznego, label=lst:genetic_algorithm_psedocode, mathescape]
Inicjuj populację $\mu$ osobników
$\textbf{powtarzaj}$
    Przeprowadź selekcję osobników z obecnej populacji
    Na podstawie wybranych stwórz $\lambda$ potomków poprzez krzyżowanie
    Na wybranych osobnikach wykonaj mutacje
    Z otrzymanej puli wybierz $\mu$ osobników
    Zastąp obecną populację nową
$\textbf{dopóki}$ warunek stopu
\end{lstlisting}
\par
Podstawowy model bywa rozszerzany o procedurę chronienia jednego lub więcej osobników o najlepszym przystosowaniu z starej populacji. Badania dowodzą, że takie podejście znacznie poprawia stabilność i powtarzalność otrzymywanych rezultatów algorytmu \cite{baluja1995removing}. Tak również postąpiono w trakcie przeprowadzania badań w ramach tej pracy - wykorzystano domyślny w pakiecie \emph{GA} rozmiar grupy chronionych osobników wynoszący 5\% populacji.
\par
Selekcji osobników z obecnej populacji do późniejszej reprodukcji osobników zwykle dokonuje się wykorzystując wartość funkcji przystosowania. Najczęściej stosowane do tego są metody \emph{rankingowa} i \emph{ruletki}. Pierwsza z nich szereguje osobniki według ich przystosowania, a następnie wybiera żądaną ich liczbę najlepszych. W dalszym przetwarzaniu wszystkie wybrane osobniki mają równe szanse na wzięcie udziału w reprodukcji. Utracona zostaje więc różnica w jakości pomiędzy najlepszymi i najgorszymi z wyselekcjonowanej puli osobników. Alternatywą dla takiego podejścia jest druga metoda. Polega ona na tym, że z pełnej puli populacji losuje się ze zwracaniem osobniki, przy czym prawdopodobieństwo wylosowania danego osobnika jest proporcjonalne do jego jakości \cite{sudholt2008computational}. Obie metody można jednak połączyć otrzymując algorytm polegający na wstępnej selekcji podzbioru populacji na którym następnie dokonuje się losowań. Przeprowadzając badania posłużono się metodą \emph{ruletkową}.
\par
O ile wyżej opisane metody selekcji osobników swobodnie można stosować niezależnie od specyfiki problemu i reprezentacji rozwiązania, o tyle działanie operatorów krzyżowania i mutacji musi zostać dopasowane do konkretnego zastosowania. Sposób kodowania rozwiązania w osobnikach wpływa jednoznacznie na możliwe operowanie nimi, a specyfika zastosowania wyznacza warunki jakie muszą spełniać nowe osobniki, by mogły być traktowane jako prawidłowe rozwiązanie (prawidłowe w kontekście dziedziny problemu, nie jego jakości rozwiązania). Dokładny opis wykorzystanych operatorów mutacji i krzyżowania zostanie przedstawiony w rozdziale \ref{ch:przyjety_model_doboru_parametrow_algorytmu}.
\par

\section{Algorytm memetyczny}
\begin{itemize}
\item rys historyczny, czym jest mem
\item wspomnienie o Lamarkianizmie i tym drugim
\item o tym że jest to rozwinięcie koncepcji algorytmu genetycznego
\item możliwe warianty memetycznego realizowane w literaturze, różne miejsca wykonywania lokalnego przeszukania
\item znaczenie parametrów algorytmu
\item pseudokod czy coś takiego
\item opis testowanych operatorów lokalnego przeszukiwania
\end{itemize}
\par
\emph{Mem} jest pojęciem wprowadzonym do użycia w 1976 w książce Richarda Dawkinsa \emph{Samolubny gen} \cite{dawkins2011}. Użył go do nazwania podstawowej jednostki w ewolucji kulturowej, czegoś na podobieństwo genu w ewolucji genetycznej. Opiera się to na obserwacji zachowania populacji, w której spotykające się osobniki wymieniają się ze sobą informacjami, przejmują dobre cechy od sąsiednich przedstawicieli lub same poprawiają się w własnym zakresie. Podstawową różnicą pomiędzy \emph{memem}, a \emph{genem} jest fakt, że \emph{mem} jest bardziej związany z fenotypem osobnika, niż z jego genetycznym niezmiennym stanem. Fenotyp w prawdzie jest w znacznej części determinowany przez genotyp, jednakże dodatkowo warunkowany jest przez środowisko zewnętrzne. W kontekście ewolucji memetycznej ugruntowały się dwa podejścia definiujące trwałość i dziedziczność cech będących skutkiem oddziaływania środowiska na osobniki: Baldwinizm i Lamarkianizm. Pierwsze z nich zakłada, że materiał genetyczny nie może ulec zmianie w trakcie życia osobnika, zatem nabyte cechy nie są przekazywane potomstwu w genetyczny sposób. Informacje o tym, jak jednostka udoskonaliła się w trakcie swojego pokolenia może zostać jednak przekazana w innej postaci, co determinuje osobny sposób kodowania cech przekazywanych \emph{genetycznie} i \emph{memetycznie}. W przeciwnym razie lepsze przystosowanie osobnika będzie widoczne tylko w postaci jego fenotypu i nie zostanie przekazane potomstwu. Drugie podejście wbrew temu co mówi rzeczywista biologia przyjmuje, że ulepszenie jednostki jest trwałe - zapisane w materiale genetycznym propaguje się na potomstwo \cite{whitley1994lamarckian}.
\par
\emph{Algorytm memetyczny} najczęściej realizuje podejście zgodne z Lamarkianizmem. Swoją budowę i zasadę działania opiera na \emph{algorytmie genetycznym} dodając do niego charakterystyczne dla siebie mechanizmy. Opisane powyżej zachowanie osobników przyjmujących dobre cechy napotkanych sąsiadów realizuje się poprzez zastosowanie algorytmów przeszukiwania lokalnego na konkretnych, zwykle losowo wybranych jednostkach. W literaturze można znaleźć przykłady implementujące doskonalenie osobników na różnych etapach działania algorytmu genetycznego. Najczęściej stosowanym jest model aplikujący przeszukiwanie lokalne przed selekcją osobników do reprodukcji \cite{costam costam} lub po wytworzeniu potomków drogą krzyżowania i przed selekcją osobników tworzących nowe pokolenie \citep{sudholt2008computational}.  


\section{Algorytm optymalizacji rojem cząstek - PSO}
\begin{itemize}
\item rys historyczny
\item symuluje zachowanie stada, roju owadów
\item pseudokod
\item opis przyjętych parametrów
\end{itemize}

\section{Specyfika optymalizacji problemów rzeczywistoliczbowych}
\begin{itemize}
\item powszechnie stosowana w wielu różnych zastosowaniach
\item konieczność spełniania constraintów na parametrach
\item niejednorodne znaczenie i wpływ poszczególnych parametrów na wartości funkcji celu
\end{itemize}

\section{Przykłady zastosowań}