\documentclass[a4paper,11pt]{article}

\usepackage{a4wide}
\usepackage[margin=3.0cm]{geometry}
\usepackage[table,xcdraw]{xcolor}
\usepackage[utf8x]{inputenc}
\usepackage[T1]{fontenc}
\usepackage[english,polish]{babel}
%%% fix for \lll
\let\babellll\lll
\let\lll\relax

\usepackage{indentfirst}
\usepackage{caption}
\usepackage{subcaption}
\usepackage{tikz}
\usepackage{amsmath}
\usepackage{amssymb}
\usepackage{hyperref}
\usepackage{graphicx}
\usepackage{listings}
\usepackage{float}
\usepackage{newfloat}
\usepackage{listingsutf8}
\usepackage{afterpage}
\DeclareFloatingEnvironment[name={Wykres }]{suppfigure}
\headsep 5mm

\usepackage{capt-of}
\usepackage{listingsutf8}
\usepackage{lscape}
\usepackage{pdfpages}
\usepackage{longtable}

\usepackage[maxfloats=250]{morefloats}
\newcommand{\skalaSchaffer}{0.5}
\newcommand{\source}[1]{\caption*{\footnotesize{Źródło:~{#1}}} }
\renewcommand{\lstlistlistingname}{Spis kodów źródłowych}

\usetikzlibrary{matrix,calc}
\usepackage{color}
\definecolor{gray}{rgb}{0.4,0.4,0.4}
\definecolor{darkblue}{rgb}{0.0,0.0,0.6}
\definecolor{cyan}{rgb}{0.0,0.6,0.6}

\definecolor{pblue}{rgb}{0.13,0.13,1}
\definecolor{pgreen}{rgb}{0,0.5,0}
\definecolor{pred}{rgb}{0.9,0,0}
\definecolor{pgrey}{rgb}{0.3,0.3,0.3}
\lstset{
  numbers=left,
  showtabs=false,
  breakatwhitespace=true,
  basicstyle=\ttfamily\small,
  columns=fullflexible,
  showstringspaces=false,
  showspaces=false,
  xleftmargin=3em,
  framexleftmargin=2.5em,
  frame=tb,
  breaklines=true,
  framextopmargin=5pt,
  framexbottommargin=7pt,
  commentstyle=\color{gray}\upshape
  inputencoding=utf8x, 
  extendedchars=\true,
  literate={ą}{{\k{a}}}1
             {Ą}{{\k{A}}}1
             {ę}{{\k{e}}}1
             {Ę}{{\k{E}}}1
             {ó}{{\'o}}1
             {Ó}{{\'O}}1
             {ś}{{\'s}}1
             {Ś}{{\'S}}1
             {ł}{{\l{}}}1
             {Ł}{{\L{}}}1
             {ż}{{\.z}}1
             {Ż}{{\.Z}}1
             {ź}{{\'z}}1
             {Ź}{{\'Z}}1
             {ć}{{\'c}}1
             {Ć}{{\'C}}1
             {ń}{{\'n}}1
             {Ń}{{\'N}}1                            
}

\linespread{1.15}
\captionsetup[table]{name=Tabela}
\newtheorem{definicja}{Definicja}[section]

\begin{document}
\renewcommand*{\listtablename}{Spis tabel}
\thispagestyle{empty}
\noindent
\hfill Wrocław, dn.\ \today\\

\noindent
\begin{minipage}[c]{0.4\columnwidth}
  Piotr Sotor, 200882
\end{minipage}

\vspace{3cm}
\begin{center}
  \begin{LARGE}
    \emph{Praca magisterska} \\
  \end{LARGE}
\end{center}

\begin{center}
  Rok akad. 2016/2017, kierunek: INF
\end{center}
\vspace{0.1ex}
\begin{flushright}
\begin{minipage}[t]{0.4\columnwidth}
\noindent
PROMOTOR:\\
Dr hab. inż.~Iwona Karcz-Dulęba
\end{minipage}
\end{flushright}

\newpage
\tableofcontents

\newpage
\section{Wstęp}

\newpage
\section{Charakterystyka algorytmów ewolucyjnych}

\subsection{Algorytm genetyczny}

\subsection{Algorytm memetyczny}

\subsection{Algorytm optymalizacji rojem cząstek - PSO}

\subsection{Specyfika optymalizacji problemów rzeczywistoliczbowych}

\subsubsection{Przykłady zastosowań}



\newpage
\section{Przyjęty model doboru parametrów algorytmu}

\subsection{Dobór parametrów części genetycznej algorytmu}

\subsection{Dobór parametrów części memetycznej algorytmu}

\subsection{Dobór parametrów algorytmu PSO}

\subsection{Badane scenariusze doboru parametrów algorytmów}

\newpage
\section{Badania}

\subsection{Specyfikacja funkcji testowych}

\subsection{Opis środowiska testowego}

\subsection{Przyjęte miary efektywności algorytmów}

\subsection{Rezultaty}

\newpage
\section{Wnioski i podsumowanie}

\newpage
\section{DODATEK I: Aplikacja - symulator}
\subsection{Założenia projektowe aplikacji}
\subsection{Instrukcja instalacyjna}
\subsection{Opis interfejsu graficznego}
\cite{lamport94}

\newpage
\begin{thebibliography}{9}
\addcontentsline{toc}{section}{Literatura}
\bibitem{lamport94}
  Leslie Lamport,
  \emph{\LaTeX: A Document Preparation System}.
  Addison Wesley, Massachusetts,
  2nd Edition,
  1994.

\end{thebibliography}

\newpage
\listoffigures
\addcontentsline{toc}{section}{Spis rysunków}
\newpage
\lstlistoflistings
\addcontentsline{toc}{section}{Spis listingów}

\end{document}